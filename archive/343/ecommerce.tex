
\documentclass{article}      % Specifies the document class

                             % The preamble begins here.
\usepackage{a4}
\usepackage{url}             
\title{E-Commerce: the future of buisness}  % Declares the document's title.
\author{Mark Angrish}      % Declares the author's name.
\date{24th March 2000}     % Deleting this command produces today's date.


\begin{document}             % End of preamble and beginning of text.

\maketitle                   % Produces the title.


\section{Computers do more than spreadsheets???}      

When people talk about computers in business, we often imagine a young
\$60K finance graduate who sits behind a terminal in a bussling office,
number crunching all day to churn out his companys profits, or, as used in
small business for letters, invoicing and spreadsheets \ldots\ slowly, the way 
we think of computers in business is changing\ldots\


\section{What is E-commerce?}

E-commerce is revolutionising the face of business all over the world.  With
an estimated e-commerce sales figure of almost \$7 billion by the end of this year\cite{Vkn:url}
, it seems not only is e-commerce here to stay, but it is
growing at an astronomical rate, and fast becoming one of the most viable means to 
promote and conduct business.  This may all be well, but what is e-commerce?

Electonic commerce (EC, e-commerce) is esentially defined as the conduct of commerce
(buying and selling) in goods and services, with the use of telecomunications, and 
telecomunications-based tools (ie. the internet, and espcially the World Wide Web)\cite{Clarke:url}.
E-commerce and e-business are often used interchangably although EB includes other
apects (eg. administration, student enrollment etc.)

E-commerce can be divided into:
\begin{itemize}
	\item E-tailing or virtual storefronts on websites with online catalogs;
	\item Electronic Data Interchange(EDI), the business-to-business exchange of data;
	\item Security of business transactions.
\end{itemize}


\subsection{E-tailing?}

A retail company's dream?  Open 24 hours, available to a global market, the ability to interact and 
provide custom information, ordering and service, and the prospect of using multimedia to sell your
product?

Surely, this is a retail companys dream\ldots\ and it exists in cybersapce.  Already a multibillion
dollar source of revenue for world businessness, the Web expands, with new businesses based entirely on
Web sales being made daily.
As concerns about secure  order-taking receded, sales escalated.
Success has been overwhelming in most cases. Dell computers reported orders of a million dollars a day
by as early as 1997.  By early 1999 projected e-commerce revenues for business were in the billions of
dollars, and stocks in adept e-commerce companies was going through the roof\cite{Builder:url}.  Simba information says 
the bigest sellers are computer products(such as Dell), consumer products, books and magazines(amazon.com)
and music(CDnow.com) and entertainment products\cite{Builder:url}.

CommerceNet/Neilson Media reported that 10 million people had made purchases on the Web, and Jupiter reseach predicted
that e-tailing would grow from \$2billion in 1997 to \$37 billion by 2002\cite{Builder:url}.



\subsection{Electronic Data Interchange(EDI)}

EDI was created by the US government in the early 1970s and now used by 95\% of
companies in the US.  EDI is the exchange of business data 
between companies that know each other well, and make arrangements for connection
via private networks, using a known protocol, or more easliy understood as the replacement
 of paper-based purchases with electronic equivalnets(computer to computer).
 Although EDI predates the modern internet, it is
now finding a role in corporate web sites.


\subsection{Security of Transactions}

Security includes authenticating business transactors, controlling access to resources(eg. Web pages for 
registered users), encrypting communications and in general, ensuring the privacy and effectiveness of transactions.
According to a survey done by CommerceNet, shoppers dont trust e-commerce as there is no easy way to pay for
things. Customers are worried about credit card theft and personal information leaks, from the 12 year old hacker
pro from the States\cite{Builder:url}.
Although Internet security breaches get a lot of press, most vendors and experts say that trasactions are actually
less dangerous in cybersapce than in the physical world.  The difficulty is getting customers to believe that
e-commerce is safe.

Ever since version 2.0 of both Netscape Navigator and Microsoft Internet Explorer, transactions have been 
encrypted using Secure Sockets Layer(SSL), a protocol that creates a secure connection to the server, protecting
the data as it travels over the internet.  SSL uses public key encryption, an extremely strong encryption methods
around\cite{Builder:url}.

Secure Electronic Transactions(SET) is quickly emerging as an industry standard, as it encodes the credit card
numbers that sit on vendors' servers so that only banks and credit card companies can read the numbers\cite{Builder:url}.

\section{ Business Has Changed}

A lot has changed in just the short time of 10 years.  Businesses that have always relyed on face-to-face customer
interaction were planning e-commerce Web sites and many business were reinventing there looks, and market.
The computer revolution that has started to encompass the globe and the technology behind computers is getting more 
sophisticated every day.  Traditional viewpoints of 
business are being redefined, long gone are the days of pen, paper and calulators, there now exists ultra fast computers
and highly sophisticated and powerful software packages that drive businesses further.  The World Wide Web has demonstated
how it can unite people and form a global economy. No longer are business trapped in there own backyard.
The use of the internet to promote business has ecouraged smaller companies to go to the web and expand their market, and
reach an otherwise unaccessable clientel
while at the same time, curtail the dominance of a larger corporation that produces similar goods or services.
This intorduces a higher level of competition in which we should see benefits for consumers.  THousands of companies
that sell products to other companies have discovered that the web provides not only for a 24-hour showcase for their
products but a quick way to reach the right people in a company for more information.  
E-commerce is here to stay and its future looks extremely bright.  



\bibliographystyle{alpha}
\bibliography{ecommerce}

\end{document}               % End of document.
