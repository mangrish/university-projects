                                 ==========
                                SQAP  Review
                                 Followup 
																============

================================================================================
FORMAL REVIEW REPORT
================================================================================

Date:                09/05/01
Reviewer(s):         sja, walkera

Group:               s340gg
Document Name:       Software Quality Assurance Plan
Document Version:    None Specified.

Date of follow-up:   16/05/01
Follow up report by: Kevin McDonald (kem)
                     Sylvia Syamanond (vss)

Follow up responses by Team G (340) will be denoted by >>

--------------------------------------------------------------------------------
REVIEW SUMMARY
--------------------------------------------------------------------------------

Overall: REWORK

* The document meets all the points outlined in the 340 manual checklist apart
from the style guide for data-flow diagrams. Only requires some slight
reworkings to the existing sections. The structure of the document is
straightforward and easy to understand and reference. There are however some
sections within the document that tend to contradict themselves such as the
directory structure and the layout guidelines as mentioned in the section
breakup.

>> These inconsistencies have been addressed and was probably a sign that
>> the document was rushed in the concluding stages. The layout guideline
>> problem arose as a result of the guidelines being created late. Also due
>> to people writing different sections. These problems have been rectified.


* The only other alteration that could improve the overall readability of
the document is the highlighting of directory and file names within the
document. Overall a good SQAP.

>> These changes were implemented in subsequent versions of the SQAP. We also
>> believed that these suggestions would improve clarity.


Sections that are missing:
* 1. style guide for diags - this is mentioned in your Verification and
Validation section for your sqap (sect 7) but there is no mention of it
anywhere in the sqap itself.

>> We no longer think a style guide for diagrams is necessary, so we have
>> deleted it from section 7.

* 2. permissions - what are your dir and file permissions like? these need to be
consistent across the group

>> Good point. We have recently added descriptions of permissions in the SQAP.

--------------------------------------------------------------------------------
DEFINITION OF RECOMMENDATIONS
--------------------------------------------------------------------------------

PASS
----
1. The review item adequately addresses the specified criteria.
2. Only minor modification is required to bring it up to scratch.
3. Modifications can be dealt with by the Section or individual and do not
require further checking.

REWORK
------
1. The review item addresses some of specified criteria.
2. The review item requires further work meet the stated standards.
3. There may be omitted information, or existing information may be ambiguous
and requires restructuring.

FAIL
----
1. The review item does not address the specified criteria.
2. The review item requires a lot of work and should undergo strict quality
assurance monitoring within the section.
3. A Formal Re-review is recommended.

--------------------------------------------------------------------------------
GENERAL CRITERIA - RESULTS & COMMENT
--------------------------------------------------------------------------------

    1. Legibility   PASS

                    The document is readable. Minor spelling mistakes, some
                    grammar mistakes also. Please highlight filenames, dir names
                    and any other important variables within the document.

    2. Conciseness  PASS

		    The document sections are to the point and concise. No work
		    required to fix this criteria.

    3. Coherence    PASS

                    Good, no work required to meet criteria.

    4. Relevance    PASS

                    Good, no work required to meet criteria.

--------------------------------------------------------------------------------
SECTIONAL RECOMMENDATIONS
--------------------------------------------------------------------------------

Title Page	PASS
----------
* Fine, maybe an outline of the content/purpose of the SQAP should be
  added to the abstract.

>> I believe we did mention the content of the SQAP in the abstract...

Section 1	PASS
---------

* Needs a clearer description of the project with better grammar.

>> We believe our grammar is OK, but you are right in saying the description
>> of the project needs to be clearer; we will endeavour to do this. 

Section 2	REWORK
---------
* Software Architectural Design Document not mentioned.

>> In the following version of the SQAP, we addressed this concern.

Section 3	REWORK
---------
* Within the role descriptions Backup Manager was mentioned twice 3.1.9
  and 3.1.11.

>> Well spotted. This highlights that our review process needs to be adjusted.
>> We have deleted 3.1.11 Backup manager.
	
* In organisation you mentioned the role of Document Manager however in
  the descriptions of the roles there was no description for the Document
  Manager, however there was a description for Document Co-Ordinator.

>> This was an error on our behalf. They are the same role which will indeed
>> be titled Document Coordinator.

* ADD mentioned in the project plan but nowhere else in the document.

>> OK. The ADD name should be replaced with SADD to maintain consistency. 

Section 4	REWORK
---------
* SQAP description needs to be clearer, what is the software engineering
  process to a client?

>> We agree, it is not very readable. The SQAP description has been reworked
>> to improve clarity.
	
* Spelling error in user documentation produce should be produced.

>> OK

* Unclear sentence also present "and also in the maintenance of the software
  that does not alter the source code". How can you maintain software with out
  altering or adding to the source code?!?

>> Yes it does not convey its meaning well. What we meant to say was  
>> administration following installation.
	
* SDD document description is actually an SADD description, and nowhere
  in this section is the true SDD document description mentioned.

>> This error has been remedied. We now have seperate descriptions of both the
>> SDD and the SADD.

 Section 5       PASS
---------
* Team Meeting Times and Location; "conducted no less the 24 hours
  prior"

* Slight spelling / grammatical errors in Task Allocation: "task as they sees
  fit" (pp 18), "directory of the teams repository" (pp 18) and "including
  [writing] any notes" (pp 19).

>> Task Allocation subsection has been completely changed so these
>> sentences no longer exist in the current SQAP. Again, this was
>> due to the lack of attention paid when proofreading the document in
>> preparation for the 341 SQAP submission. 

* Document Standards mentioned that indentation be only used for pointed
  lists and that paragraphs be separated by a single blank line. Most of the
  document was inconsistent with these two guidelines. Paragraphs where
  indented sometimes and rarely separated by a line except in later sections.

>> This was due to several team members proofreading different parts of the
>> SQAP. Proofreading process was refined after this submission. 

* First sentence of Coding Standard should be made clearer and more
  grammatically correct.

>> This subsection undergone a change after the internal review so this
>> has been corrected.

* The group directory in coding standard should be re-aligned on the next line.

>> The external reviews by the 440 students pointed this out and has been
>> corrected accordingly.


Section 6       PASS
---------
* The times at which the audits should take place should be mentioned in
  relation to the phase of product development. Mentioned in some, not in
  others.

>> This was also brought up in the external reviews by the 440 students.
>> Again, this careless mistake was due to a lack of proofreading. Team
>> G learnt a valuable lesson from the SQAP submission and had since
>> been improving on our proofreading. 

* In-Process Audits personnel involved not mentioned.

>> This had been fixed up during the internal review following the
>> 341 SQAP submission. 

Section 7       REWORK
---------
* In validation criteria style guide for data flow diagrams is mentioned
  but not present within the document.

>> Data flow diagrams will not be used and this has been taken out
>> of the current SQAP. 

* Preliminary Software Design Description mentioned however not present in
  milestones or reference documents. Should it be renamed as ADD or even
  SADD?

>> This was referring to the Software Architectural Design Description. We
>> now refer to this as SADD and keep this consistent throughout the SQAP.

* Acronym SDD used to describe both the Preliminary Software Design Description
  and the Detailed Software Design Description, may cause confusion.

>> There was confusion over the difference between SADD and SDD. This has been
>> resolved and the SQAP was updated so that SADD and SDD are distinguished
>> from one another. 


Section 8       PASS
---------
* Fine.

Section 9       REWORK
---------
* Title capitalisation to be consistent with rest of document.

>> Again, these mistakes were detected and corrected after internal review.

* Fix first sentence of Design Methodology. "The Waterfall model will be used as
  a basis for the design method we will be using the waterfall model" ?!?

>> This revealed the lack of proofreading on the SQAP. It was fixed
>> during the internal review.

* CVS structure uses directory naming that does not conform to the standards
  specified earlier in section 5.5. ie. Design-notebook should be
  Design_notebook.

>> This was pointed out during the external reviews by 440 students and
the SQAP was fixed up and the Librarian renamed this subdirectory in our group 
directory so that it conforms to the standards.

* Changes from third person to first person in records collection, maintenance
  and retention.

>> The internal review revealed this inconsistency throughout the SQAP. 
>> The SQAP Document Coordinator and Review and Audit Manager have gone  
>> the whole document and made sure this problem is fixed and that all terms
>> used are consistent.

* No Bugs directory mentioned in directory structure (pp 31) but referenced in
  Section 8.2

>> Internal review revealed this problem and has since been fixed.

* No Risks directory mentioned in directory structure (pp 31) but referenced in
  Section 10.6

>> Same situation as the Bugs directory problem above.

* Separate directory for Agendas in group directory structure however agenda for
  each meeting is placed in the /Design-notebook/Meetings/ directory in the
  Meeting Agenda section (pp 17).

>> Group directory has undergone major changes since the initial 341 submission
>>of the SQAP. We now have sections for agendas for different types of meetings.
>> Agendas for team meetings are found under the group directory:
>> /Common/Design-notebook/Meetings/Group/Agenda.
>> while agendas for client meetings are found under the group directory in:
>> /Common/Design-notebook/Meetings/Client/Agenda

Section 10      PASS
----------
* Fine.

-------------------------------------------------------------------------------
END OF FORMAL REVIEW REPORT
-------------------------------------------------------------------------------            
                                 ==========
                                 SRS Review
                                  Followup 
																 ==========

================================================================================
FORMAL REVIEW REPORT
================================================================================

Date:                10/05/01
Reviewer(s):         dcgl, sss

Group:               s340gg
Document Name:       Software Requirements Specification
Document Version:    1.21

Date of follow-up:   17/05/01
Follow up report by: Flynn Connor (frc)

Follow up responses by Team G (340) will be denoted by >>

--------------------------------------------------------------------------------
REVIEW SUMMARY
--------------------------------------------------------------------------------

Overall: FAIL

This document seems extremely rushed, and it shows. There is by no means any
information that would help the client understand the project better. A lot
of programming/implementation content has been included which should be left
to the SADD/SDD after a design process has taken place. Otherwise it could
be used against you in a court of law. If we were the client, we would not sign
this document. We may even consider hiring a new supplier based on this
document alone.
>> More details have been included that were not available at the time the
>> document was created. Legal issues are not considered relevant enough to
>> include in this third year document, since legal terms would possibly
>> cause more confusion than is necessary for this level. It would however
>> addressed in real world contracts, etc, that is beyond the scope of this
>> project


An (early) description of the different types of 'users' would help remove any
confusion associated with the use of these terms. Also, greater use of formal
methods (eg. use case, use case diagrams) would greatly increase the
readability when describing the proposed system.
>> Users have been described better. Diagrams havve been improved.


Overall you really need to make some major revisions to this document
before it can resemble a proper SRS or legally binding document.
>> It is agreed that more work is necessary and has been undertaken to make
>> this document better since this version, but the points made in this review
>> similar to those discussed by previous reviews and were still relevant in
>> some cases.


Sections that are missing:
1. acceptance criteria - one sentence does not constitute a section. Read the
detailed section description for reasons why.
>> see comments below

2. appendix - should also include an appendix for things like changes to the
SRS throughout the year. useful to include them in here because you will be
changing and adding requirements.
>> No appendix was considered necessary at this stage since there was nothing
>> to put in to it

3. client signature - blank page with a title does not count as a section.
>> see comments below

Although the verdict is fail, I think you should definitely internally
review the SRS after modifications. The standard really needs to be improved
if this is ever to resemble a legally binding document.
>> This process has been completed and resulted in a document that we belive
>> to be superior to the one that was reviewed.

--------------------------------------------------------------------------------
DEFINITION OF RECOMMENDATIONS
--------------------------------------------------------------------------------

PASS
----
1. The review item adequately addresses the specified criteria.
2. Only minor modification is required to bring it up to scratch.
3. Modifications can be dealt with by the Section or individual and do not
require further checking.

REWORK
------
1. The review item addresses some of specified criteria.
2. The review item requires further work meet the stated standards.
3. There may be omitted information, or existing information may be ambiguous
and requires restructuring.

FAIL
----
1. The review item does not address the specified criteria.
2. The review item requires a lot of work and should undergo strict quality
assurance monitoring within the section.
3. A Formal Re-review is recommended.

--------------------------------------------------------------------------------
GENERAL CRITERIA - RESULTS & COMMENT
--------------------------------------------------------------------------------

    1. Legibility   PASS

                    The document is readable. Minor spelling mistakes, some
                    grammar mistakes also.

    2. Conciseness  FAIL

                  Things like the Requirements section is not very concise.
		  What makes it worse is the fact that you don't even address
		  the issues related to the section.

    3. Coherence    REWORK

                  There is little reference to other sections within the
                  document. Very little teamwork between the collaborators of
		  the document. There is also little reference to other
                  documents. Although structuraly sound, there is very little
		  flow of ideas within the document itself.

    4. Relevance    FAIL

                  This document is an SRS. Not a SADD/SDD. You have not
		  addressed the main aims of specifing the "requirements" as
		  elicited from the client. You should not mention "how" the
		  criteria are to be met, only "what" they are within this
		  document. Programming language and other technical issues
		  are best left to other documents.

--------------------------------------------------------------------------------
SECTIONAL RECOMMENDATIONS
--------------------------------------------------------------------------------

Title Page		REWORK
----------
* "Training Works" - could give a more descriptive name to the project.
>> The title given to the project from the client was "TrainingWorks", so this
>> suggestion isn't relevant

* should state who the proposed system is for.
>> This has bee done with the client name and some details

* the Abstract clearly states the intentions of the document.


Section 1		REWORK
---------
* "...existing system the client already has in place.", poor grammar, a
  tautology.
>> All grammar through-out the document has been modified

* mention of 'users' however no description of who they are and what their role
  within the system is. Outline the people (administrators, trainees ...)
  involved in the system here.
>> The system was to be defined in terms of the capabilities that individuals
>> would have, rather than predefined "roles". This point has been explained
>> better in more recent versions

* better flow if issues are mentioned in order they will be addressed in the
  document. For example, mention existing system before Training Works.
>> Structure has been modified to reflect this

* should have a more specific outline of document structure.
>> A purpose has been included to discuss aspects like this

Section 2		REWORK
---------
* should be broken down into subsections, to aid readability.
>> This has been done

* a visual aid and possibly use cases would help with written descriptions.
>> more examples of the data used has been included to explain them better, but
>> since the data flow is direct to and from the one point, DFD's were deamed to
>> be a waste of resources

Section 3		REWORK
---------
* (Last sentence of 3.1) talks about the users but still no mention of who
  they actually are. Should put a more detailed description at the beginning
  of the section, because this makes it easier for whoever reads the document
  to understand the concept of 'user'
>> users have been defined better, both in early sections and in this section

* (section 3.2) "Shaded" should not be capitalised.
>> This diagram has been removed since it was decided that it did not display
>> effectively what it was meant to.

* (section 3.2) ".. regions of the diagram .." should reference diagram.
>> see above comment

* title of Figure 1 not descriptive enough. Control Diagram of what?
>> see above comment

* brief, should go into further relevant detail of the system.  For example
  could use a data flow diagram to show interactions with the database.
>> again, DFD's would not be effective, in that they would only show one
>> stream of input data, and one stream to a user coming from the system


Section 4		FAIL (BADLY!)-
---------
* This section goes into programming details (eg 'OwnRecord, EditAll..') when it
  should only have a description of the requirement not how the requirement is
  to be met. If the aforementioned variables, functions do not exist in the
  final product then this could lead to legal issues because they form part
  of the acceptance criteria.
>> These were client required capability variables, therefore they do form
>> part of the requirements, they also give names to the capabilities mentioned
>> that were used through-out the rest of the document

* should be a brief discussion of what functional requirements entail.
>> This was considered implicit in the name of the sections

* (section 4.1) introduction of Core Requirements needed, could mention that
  they form the basis of acceptable criteria.
>> This was mentioned in the section specifically called "Acceptance Criteria"

* (section 4.1.1) Still no mention of user. How do they get "appropriate
  capability" ? What is capability in reference to.
>> see above for users comment. This was an outline of what capabilities are
>> in the system before they are referenced to what was their specific role in
>> the system. This section now reflects what sections each capability is
>> referrenced in.

* (section 4.1.1) "Edit non-user records" who are the non-users. Why do they
  have records?
>> Non-User Records have now been beeter defined. This actually meant records
>> kept about institutions, courses, and modules.

* (section 4.1.1) it is unclear exactly what the user is editing when they "Edit
  user profiles, hence capabilities."
>> Now referenced to the section describing the functional activity of editing
>> user profiles

* (section 4.1.1) a clarification of the statement, "These capabilities will be
  separate, so that a user may have any combination from an available list."
  is needed.
>> This has been reworded to attempt to make it clearer.

* (section 4.1.2) subsection heading had no relevance whatsoever to the info
  contained in the subsection. The main sentence is phrased badly.
>> All headings for the "Functional Requirements" sections have been changed
>> to be clearer.

* (section 4.1.3) subsection heading had no relevance whatsoever to the info
  contained in the subsection. The main sentence is phrased badly.
>> see above comment

* (section 4.1.3) heading states "users may have their own record", but talks
  about editing their own or others records, should use a more descriptive
  heading.
>> see above comment

* (section 4.1.3) "An text.." should be "A text..".
>> gramatical errors have been fixed

* (section 4.1.3) it is unclear what type of user has a record on the system and
  "EditAll capability".
>> see above comments about users profiles, rather than "roles"

* (section 4.1.4) subsection heading had no relevance whatsoever to the info
  contained in the subsection. The main sentence is phrased badly.
>> see above comment

* (section 4.1.6) "This process is to be performed off-line so no output to the
  user is expected in the interface".  Does this mean the "user" can not find
  out their marks for a given module?
>> This section relates only to skill level calculations, not other user
>> details

* (section 4.1.6) Not a functional requirement. The section heading is, however
  the description is not.
>> This has been changed

* (section 4.1.9) What? Needs more clarification.
>> this has been given

* (section 4.1.9-10) Even though not relevant to this section, you do not
  mention the level which is needed to access these.
>> These are accesable to all users of the system. This has now been stated

* (section 4.1.13) what is a "non-modal"?
>> this has been placed in the glossary

* (section 4.1.13) who are the "non-users"?  "non-users" still has not been
  defined. Use cases would be useful in describing the users.
>> see above comments about what non-user records are

* (section 4.1.13) "Heirachical indexed browsing technique" wtf? could your
  final product stand up to this in a court of law?
>> This has been modified

* (section 4.1.13) No labelling of the figure. What does it show?
>> This has been modified

* (section 4.1.14) Even though not relevant to this section, you do not mention
  the level which is needed to access these.
>> This was an oversight, it has been modified

* (section 4.1.15) "all the capabilities set for an administrator" What are the
  capabilities of an administrator? If its not called Admin in the final product
  you may be sued. Once again this is not relevant in this section!
>> This has been better explained, and is was a functional requirement that
>> a user profile be available to use the system with all capabilities

* (section 4.1.16) "set of default capabilities" what are the default
  capabilities. We feel once again that this section is not relevant in
  Functional Requirements.
>> Default capabilities are to be changable, so there is no specific constant
>> default. Again this was a functional requirement.

* (section 4.1.16) first mention of a Guest. Who are they? What can they do?
>> This is a name to be given a user profile that is to be provided. It has
>> been better explained

* (section 4.2) What are non core requirements? To the client reading this what
  does it mean, are they not going to be implemented?
>> A description has been added

* (section 4.2.2) first mention of pre-requisite modules, what are they?
>> They were mentioned in section 4.1.4, and described

* (section 4.2.4) "Adacel system password" what is it? where does it come from?
>> This has been included in the glossary

* This whole section is very ambiguous, filled with discrepancies, wordings open
  to interpretation. Even though we have no legal background we are certain that
  this section would be your downfall because of the aforementioned factors. You
  should not put programming definitions inside this section as it forces you to
  implement this legally, before any real design process has taken place. We
  cannot mention enough times that this is the main legally binding document
  between you and the client, and as such needs to be less open to differing
  interpretations. You seem to have somewhere lost the purpose of this section
  which is to outline the requirements that have been established though
  mediation with the client. Not how they will be met but rather what they are.
>> All requirements have been expanded upon to reflect not only the actual
>> functionality required, but also the technical details required to be
>> implemented. The client has given clearer requirements, that not only
>> the functionality, but also go into the design of the functionality, and
>> rather than making many points in the design constraints sections and trying
>> to link them back to functional requirements, it was decided to include them
>> with the actual requirements.

Section 5		REWORK
---------
* description of non-functional requirements should be added.
>> This has been added

* (section 5.1) have to discuss in more detail users with "edit capabilities",
  general users and non-users, at some point in the document because these
  issues are still unclear. This section should be referenced in previous
  sections, as it partly explains who the users are. Will need more elaboration
  to be fully describe users.
>> see above comments

* (section 5.3) How does this effect the overall system? limitations of the
  hardware/software?
>> This is the required platform from the client. The affect of these on the
>> system are to handled by the design of the system

* (section 5.3) wtf? what do these names mean to the client? They are not
  referenced in the glossary.
>> They were given to the design team by the client. We were the ones who they
>> meant nothing to, so we were required to look them up

* (section 5.4) what are these "other existing databases"
>> There are many other databases controlled by the client, but any linking to
>> these would be handled by the client as required. This has been explained.

* (section 5.5) "not take longer than one second to respond" is this a tad
  optimistic even in an interactive real time system. They will hold you to
  this figure!
>> This has been clarified by the client, and relaxed. It is however a guide
>> expected response time. This has been modified.

* (section 5.5) and (section 5.6) there should not be any TBA's (to be advised),
  for example "The quality of the user interface is to be determined by the
  client upon description."
>> This is to be developed through-out the project so that the client is more
>> involved with the final outcome of the product. This has been addressed.

* (section 5.6) all the specified quality requirements are not measurable, and
  will differ from person to person. Maybe they should not be here.
>> These are standard definitions for quality attributes. They have been
>> elaborated upon.

* (section 5.7) what is a JDBC? what is mySQL? why are they not mentioned in the
  glossary?
>> These terms have been added to the glossary

* (section 5.8) What is JSP?
>> see above comment

* (section 5.9) How does gaining access to 'data' concern "another user"? What
  is the "global password system" ? Aren't the capabilities already mentioned
  and defined in the Functional Requirements? (eg OwnRecord, EditAll etc) even
  though they should not be there!
>> It is clear that the capabilities have confused the issues, but they in fact
>> relate to the functionality of the system. Access to the system as a whole
>> is governed by a combination of user profiles and the client's global
>> password system. This has been addressed through-out the document to try
>> to clear up this mis-conception.

Section 6		REWORK
---------
* A description of what the user interface is should be added. Not just what
  the user interface contains.
>> This was considered implicit in the section heading

* Need more information on the whole User Interface.
>> What kind of information are you referring to? More details have been
>> included through-out the whole document

* Grammar - 'an in Active Window' should be 'in an Active Window'
>> thanks

* What is a "main menu frontpage" no mention of this so far.
>> more descriptions and screenshots have been given

* final dot point should have been included earlier.
>> ordering of points have been modified to reflect their relative priority

* Figure 2 - Would improve understanding of product if a more detailed diagram
  of the interface is included.
>> screenshots and detailed descriptions have been added

Section 7		PASS
---------
* Surely not all the users need the same 'tutorial' documentation. An
  Administrator is a 'user' as defined by  the document. Is she/he supposed to
  administrator the system using the 'tutorials' as a guide.
>> The system is to be designed in a way that the help documentation will be
>> most useful in this format. The client requested that tutorial type help
>> documentation would be provided

* Last paragraph; how does the administrator know how to run and setup the
  system without documentation?
>> This has been clarified by the client, and only installation guides were
>> required in hard-copy format

Section 8		FAIL
---------
* What is the point of having a whole section if it is just one sentence? This
  should contain a summary of your Core Requirements alongside any other
  stipulations not previously mentioned.
>> This has been modified to reflect these suggestions

* Mention that these are the legally binding criteria upon which the finished
  product will be compared to in the case of any legal proceedings, pertaining
  to the specified product meeting the requirements of the client.
>> This is a little to technical for a third year project. Any legal references
>> would be made up since we have no access to legal advice, and would more
>> than likely cause us more legal trouble than it would be worth. It has been
>> acknowledged however that this kind of information would be included in a
>> document of this kind in the real world situations.

Section 9		REWORK
---------
* Formal methods would help to clarify this section. (eg. Use Cases, Use Case
  Diagrams). No specific technique has been used to describe the behaviour of
  the system, rather than straight paragraphs in sequential order.
>> better use case scenarios have been included. Use case diagrams and other
>> related Object-Oriented Analysis techmiques have been applied in the SADD

* In section 9.2 Viewing Records "..user is either enrolled or has.." should
  be ".. user is either enrolled in or has ..".
>> thanks

* (section 9.2) first reference of "icon bar" is this the "tool bar" that
  contains icons?
>> yes, this has been changed

* (section 9.3) "identifiable by a cursor" not very specific. A cursor is always
  present on a screen, are there multiple cursors at one time? Does it change
  in appearance?
>> the screen shots now show more information, as well as the "User Interface"
>> section having more details

* (section 9.3) "upon selecting the okay option" what does this do? What is
  okay? What does it refer to? Don't you mean 'ok' button? If I select it will
  everything be okay(jk)?
>> These kind of issues have been addressed in modifications through-out the
>> document

* (section 9.3 - 4) What is a "combo box"?
>> this has now been included in the glossary

* (section 9.4) Should include diagrams for clarity as to what the layout will
  entail.
>> screenshots are now provided

* (section 9.4) What are "browsing window" and "main window"
>> "User Interface" and screenshots now define these two better

* In section 9.3 Editing Records "To delete record,.." should be "To delete a
  record".
>> thanks

Glossary		FAIL
--------
* the Glossary is sparse. It should include all Acronyms eg. JSP, all
  application names eg. Excel and terms that are ambiguous eg. non-user.
>> More terms have identified from this review have been included, as well as
>> other terms that were identified as possible point of cofusion in our own
>> group review

Client Signature	FAIL
----------------
* A description of what the client is signing and why is paramount. The client
  will not sign this until they know what they are signing, if they do then get
  them to sign a check for you as well(jk)!
>> Again, this is only a third year document. Any legal jargon would just cause
>> us more hassle and land us in more trouble in the future. It would be
>> required in larger scale projects

* You also need to have the date, and the signature of any witness' as well.
  Ideally, you would like two client signatures from two different people, and
  another two from team members.
>> lines for multiple signatures, and date fields have been included
--------------------------------------------------------------------------------
END OF FORMAL REVIEW REPORT
--------------------------------------------------------------------------------

                                 341 Review
                                 ==========
                                 SADD Review
                                 ==========

================================================================================
FORMAL REVIEW REPORT
================================================================================

Date:               10/05/01
Reviewer(s):        zbz, tdyu

Group:              s340gg
Document Name:      Preliminary Software Design Description
Document Version:   1.36

--------------------------------------------------------------------------------
REVIEW SUMMARY
--------------------------------------------------------------------------------

Overall: FAIL

This document appears extremely rushed. There is limited 
information that would help the programmer clearly understand the project.
�This project was rushed to meet the deadline for 341, and was not up to the
�standard we usually expect.

Overall you really need to make some major revisions to this document
before it can resemble a SADD of the standard that can be used in a real project.
>>Work on fixing this document began pretty much as soon as it was submitted.
>>We understand the need for a high quality SADD in order to complete our project.

Sections that are missing:
Was not really any sections missing that could be determined. Rather the main 
problem lay the content of each of the sections. 
>>We put in the right sections, and your right, content is a little sparce.

The standard really needs to be improved
if this is ever to resemble a useful design document. A measure of this is that
a programmer who was not part of the design process up until now, should be able to
pick up the SADD and understand the top-level design of the system. After
repeated re-reads of the SADD, the system design was still unclear hence not passing 
this measure.
>>We are already undertaking extreme action to rectify this document. We understand 
>>its importance to developing a quality project.  I believe however, that a programmer
>>should be able to see the way we did our design fairly well, but not the actual 
>>content.We have put in place a big strategy however, to make sure this document is
>>done to the highest level of quality possible.

--------------------------------------------------------------------------------
DEFINITION OF RECOMMENDATIONS
--------------------------------------------------------------------------------

PASS
----
1. The review item adequately addressees the specified criteria.
2. Only minor modification is required to bring it up to scratch.
3. Modifications can be dealt with by the Section or individual and do not
require further checking.

REWORK
------
1. The review item addressees some of specified criteria.
2. The review item requires further work meet the stated standards.
3. There may be omitted information, or existing information may be ambiguous
and requires restructuring.

FAIL
----
1. The review item does not address the specified criteria.
2. The review item requires a lot of work and should undergo strict quality
assurance monitoring within the section.
3. A Formal Re-review is recommended.

--------------------------------------------------------------------------------
GENERAL CRITERIA - RESULTS & COMMENT
--------------------------------------------------------------------------------

    1. Legibility   PASS

                    The document is readable. Minor spelling mistakes, some
                    grammar mistakes also.
		>>These have since been corrected.
    2. Conciseness  REWORK

                    The document was too concise, skipping explanation of many 
				  parts of the design. Hence making understanding such things as 
				  diagrams very difficult.
		>>Design was explained adequatly enough we felt, but justification, was
		>>not done very much, this has also been rectified.
    3. Coherence    FAIL

                  There is little reference to other sections within the
                  document. There is also little reference to other documents. 
			Although all relevant sections appear, there is very little
		  	flow of ideas within the document itself. Such things as putting
			all diagrams in the appendix and no discussion did not help matters.
		>>We are still working on a way to fix our diagrams, and their positioning.
		>>We have since gone back, and tried to reference as many of our decisions
		>> as possible, and an awareness, for future expansion of this document.

    4. Relevance    PASS

			For the most part all the sections covered in the SADD were
			relevant to the design. However, decisions such as  the size
			of arrays should be left to the SDD, eg in the Detailed
			Class Diagram. This Document should only concentrate in the 
			module breakdown and design.
			>>No i disagree, we followed Schach, and it clearly shows to place
			>>string and int lengths in the arch design as well as requirements.
			>>We are alsoe pretty sure our document concentrates on module breakdown
			>>and design, after all this is an architectural design document.
--------------------------------------------------------------------------------
SECTIONAL RECOMMENDATIONS
--------------------------------------------------------------------------------

Title Page		REWORK
----------
* the Abstract  does not clearly state the intentiones of the document. eg That this 
  document's target audience is the programmers. That it should help them understand 
  the top-level design of the system.
>>This documents target audience, is the team, the client and any stakeholders. 
>>It also says that this document describes the arch design and the database design.

* Titling of the document is confusing. First it refers to the document as a
  "Preliminary Software Design Description", then as a "SDD" and the expected was
  "SADD". The team should really decide what standard titles should be used and 
  stick to them. As well the description of this document is completely missing from
  the SQAP which maybe the reason  for the confusion.
>> Good point, we'll get on that!

Introduction		PASS
------------
* section 1.1 
   * Spelling error - excel
   * Indicate Windows as a registered trademark.
   * Paragraph 2, sentence 1 - Difficult to read.
   * Paragraph 3, sentence 1 - Bad reference to Schach
>> We have taken  the above advice onbaord.
* section 1.5 
   * Paragraph 1, sentence 1 - Bad reference to Schach

* section 1.6
   * Should fix TM (trademarks)
   * What is JDBCTM? What is version number x.x?
   * What is mySQL?
>>These TM signs have since been rectified.
>>the reference to mySQL has also been rectified.
Design Description		FAIL
------------------
* Hyphenation will help readability, eg. non-core, use-case.
* Should place diagrams in with the relevant section, not in the appendices.
>>We agree, and we are working on fixing these problems.

* section 2.1 
   * Maybe explain that OO Analysis is broken into three parts, being USE CASE
   * MODELLING, CLASS MODELLING & DYNAMIC MODELLING and describe what they are 
	 there instead of where the modelling actually occurs. 
>>We planned to have a breif overview, but ran out of time, we will put this in.
   * 
* section 2.1.1
   * Use-case diagram should be included in this section. 
>> Trying to fix this problem with pic positioning.
   * Maybe have a brief discussion of each of the use-cases before going into detail.
>> We have since decided upon branch trees, to help us get better clarification 
f what is going on.
   * Typos - each (not Each), misplaced comma. 


* section 2.1.2 
   * Initial Class Diagram should have been placed here and a description of
     each of the classes would help with explaining their function in the
     system. This would have helped in giving an overview of the system design.
>>Yes we know, see above about Use case diagram.
	 A suggestion, the  process of noun extraction and iterations of class diagrams 
	 could have been put into  appendix to show how this class diagram was arrived at. 
>>We like this suggestion, and plan to follow it up.
   * Also it makes reference to noun extraction done in the SRS. The copy of the SRS 
     that we received did not contain noun extraction.
>>see SRS review.
   * No blank line between first, second and third paragraph, or bad spacing between
     sentences.
   * Bad reference to Schach.(BiTeX didn't work)
>>still in process of fixing.

* section 2.1.3
   * State Diagrams should have been placed here and a brief discussion to put
     them into context of the design. Then put the second interaction of the
	 class diagram with the methods that have been derived.
>> Yes we realise, but they were omitted due to time constraints.
>>We agree with the placing of a discussion for context.
   * No blank line between paragraphes.(As specified in the SQAP)

* section 2.1.4
   * Admitting that you have not done any formal testing is never a good thing!!!!!
>>We had no time, we plan to do it still.
   * No blank line between paragraphes.

* section 2.2

* section 2.2.1 
	* The user-cases are not scenarios so they should not be used as the basis
	  the sequence diagrams. Rather scenarios should be derived from the
	  user-cases.
>>Umm.. we didnt use use cases as scenarios...they were derived from the the
>>use cases, as demonstated in Schache.
	* The sequence diagrams should be placed in this section with the scenarios
	  that have been created from the use-cases, making it very clear how the
	  the objects in the system interact. 
>> This is a good point, and we will follow it up. However it already makes 
>> reference in the SADD, that sequence diagrams, came from the scenarios.

* section 2.2.2
   * Detailed class diagrams should be placed here and explained.
>> See above, about use case diagram
   * Paragraph 2, sentence 3 - Class diagram (singular)? There are two diagrams.
>> It was split to fit the page....
   * Detailed Class Diagram should be placed here and additions to classes
   * should be noted.
>>We agree, and will work on it.

Object Designs		PASS
--------------
* section 3.2
   * Typo on StudyRecord

Database Structure		PASS
------------------
* Paragraph 1, sentence 2 - grammatical error.

Appendix A		FAIL			 	
----------
* Use case diagram:
   * Use case diagram describes the interaction between the user and what?
>>The omitted work screen, should have been in there, it now is....
   * Are there three systems the user is interacting with, or is there one?
>>One System, different modal windows, hence 3 boxes.
   * Lines from user do not touch ellipses. The purpose of the Use Case Diagram
     is to show the System's Use Cases and which actors interact with them and
     not the interactions between systems.
>>I have no idea what your talking about here...but other use cases, spawn other
>>use cases, thats why some lines make others.

* Class Diagram (top level):
   * Diagram is ambiguous. CLASS DIAGRAMS derived on the class modelling are 
     very unclear. The class diagram should show the top level existence of
     classes and their relationships in the logical view of the system. The
     class diagram presented, seems to indicate that all the parts of the 
     system is disconnected from each other.
>>Disreguard that diagram, it was formulated very quickly for the deadline,
>>and not a reflection of the real work we have put in.

* State Diagram:
   * The state diagram is used to elicit the methods of each class. 
   * Use the dot at the tail of the default arrow to indicate Login state as the
     default state.
>>We know, and it was ommitted due to time constraints, and general stress.

* Class Diagram:
   * Very cluttered - Difficult to read a relationship from the diagram and
     identify which classes it relates to.
   * Too detailed - Fields and sizes should not be considered yet. Leave these
     details for the detailed class diagram.
>>See above class diagram response for point one, and we disagree with point two.
>>Refer to Scache, as this is the model we followed.

Appendix B		FAIL
----------

* Section 1.5 states that UML notation has been demonstrated throughout. In UML,
  an instance of a class (an object) is represented by the underline name of
  the class in lowercase, and only objects are shown in sequence diagrams.
  Clearly, UML notation was not followed.
>> Ouch...take it easy champ, you could have just told us to fix this error...
>> which we are still in the process of rectifying anyway.

* Sequence Diagrams would be much clearer if a Scenario List was included before 
  the diagram. Most examples seen in textbooks have each stage in the sequence
  numbered. These numbers correspond to the numbers in the scenario list (see
  Schach) As well there appears to be more than one scenario on each sequence 
  diagram making them very hard to follow. This maybe because the use cases
  were used instead of creating of scenario from them.
>>We agree with the first point, however, due to the shear number of possibilities
>> that this project encounters, it is hard for us to just define one scenario
>> from a use case, so we EXPLAINED, why we used more scenarios, so we can illict
>> more sequence diagrams, and use them to pick  up and methods we may have missed.

* Each message between objects relate to one step in its scenario, and each step
  in a scenario relates to one message between objects. I can not see
  any consistency in this respect, even though you haven't get scenarios. If
  this was consistent, the several alternatives makes this very difficult to
  follow.
>>We admit the fist lot of sequence diagrams were bad, and are being rectifyed
>>this weekend.

* Sequence Diagram for the Login Scenario:
   * You can not use "other" as one of your objects. Schach can use "other"
     because his Art Dealer example has a class called "Other".
>>see above point.

* Sequence Diagram for the MyRecord Scenario:
   * Spelling errors - lgin, r, MylRecords.
>>see above point.

* Sequence Diagram for the AllRecords Scenario:
   * Training History is not one of your classes.
   * Spelling errors - editahle, inclue.
   * There are two interactions without description.
>>see above point.

* Sequence Diagram for the Browse Scenario:
   * Institution, Course, Module and Skill are not classes within this project.
   * Some interactions are badly labelled or even not labelled at all.
   * Typo - Servelet.
>>see above point.
Appendix C		REWORK
----------
* Diagrams take up a whole page, yet, page numbering does not reflect this.
>> We are still working on a way to rectify tis problem
* Detailed Class Diagram (top level):
   * This diagram included twice on two consecutive pages?
   * Relationships between entities are missing.
   * typos - BrowseServet, DeleteContent().
>>see the last two points...
* Detailed Class Diagram:
   * Missing?
>>no it was included as a hardcopy unless you didnt recieve one.
Appendix D
----------
* These don't look like Entity-Relationship diagrams. To what section do these
  diagrams belong to? Section 2.2.1 makes no reference to these diagrams.
>>We are still redoing our original database design

* E-R Diagram for the User Profile database section :
  Looks like it is in Appendix C.
>> Does it? okay then.... We have fixed up our document structure so that appendix
>> no longer exists.
References
----------
Incorrect use of quote marks?
Typo - misplaced comma in first reference.
>>Again this was a rushed document, so this will be fixed.
________________________________________________________________________________

Attachments
-----------
No part of the SADD made reference to Attachments.

>>We had no need at the time, but now we do, after a few revsions of this document.

Appendix A in Attachment
------------------------
Repeating one portion of Appendix A.
>>We noticed it too.

Appendix C in Attachment
------------------------
Looks like a mix up of pages with Appendix C. This contains two identical
    detailed class diagrams, not included in Appendix C.
>>Again, this was very rushed...have mercy on our souls =P...

--------------------------------------------------------------------------------
END OF FORMAL REVIEW REPORT
--------------------------------------------------------------------------------
