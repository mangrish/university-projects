\documentclass{article}

\usepackage{epsfig}

\title{433-444 Software Reliability and Testing\\
	Assignment 3 - Safety Analysis\\
	}
\author{Mark Angrish (mangr)\\
  Student Number: 74592}
\date{\today}

\begin{document}
\pagenumbering{roman}

\maketitle

\begin{abstract}
\noindent
This document describes the processes, findings and conclusions, of performing
preliminary hazard analysis and fault tree analysis for a motor manufacturers
computer assisted braking system (CAB).
\end{abstract}

\newpage

\tableofcontents

\newpage

\pagenumbering{arabic}
\section{Introduction}

\subsection{Aim of project}
The purpose of this document is to demonstrate the process of conducting
preliminary harard idenification and analysis, followed by a fault tree analysis
for a major motor manufacturers computer assisted braking system (CAB).

\subsection{Overview of Safety Analysis Process}
The safety analysis process was conducted as follows, incorporating what
activities are to be performed in each step:

\begin{enumerate}
	\item \textit{Preliminary Hazard Analysis}\\
	\noindent
	PHA consists of two parts, that is:
	\begin{enumerate}
		\item textit{Preliminary Hazard Identification}\\
		\noindent
		is aimed at determining potential hazards for a system in a group 
		based activity.\\
		\item \textit{Preliminary Hazard Analysis}\\
		\noindent
		is aimed at determining potential hazards for a system as well as
		well as the causes of failure leading to these hazards and the 
		liklihood of such failures.\\
	\end{enumerate}	
	\item \textit{Fault Tree Analysis}\\
	\noindent
	is a deductive technique for hazard analysis, and is drawn by working
	back from hazards to basic causes.  Fault trees are a graphical notation
	and consist of events and gates.\\

\end{enumerate}

\newpage
\section{Preliminary Hazard Analysis}

\subsection{Preliminary Hazard Identification}
\noindent
A preliminary hazard identification brainstorming activity session was 
conducted informally by a group of 3 people.\\
Initially, the team created a checklist of weather conditions, road types,
and random possible triggers.  We used this checklist to ensure completeness
of the hazard analysis.  The checklist used appears in table 1.\\

\begin{table}[h!tbp]
\begin{center}
\begin{tabular}{|l|l|l|}
\hline
\textbf{weather conditions}	&	\textbf{road types}	& \textbf{random triggers}\\
\hline
rain affected- heavy&	smooth bituman/concrete&	weight of car\\
rain affected- light&	rough bituman&		uphill/ downhill\\
hail&	gravel	&	speed\\
sleet&	mud&\\
snow& 	dry mud/dirt&\\
dry&	soil/grass&\\
ice/frost&	&\\
\hline
\end{tabular}
\end{center}
\caption{Preliminary Hazard Identification session checklist}
\label{fig:PHIChecklist}
\end{table}



\noindent
After generating these conditions and types etc. one varibale was kept constant
while conditions wer eiterated through.  For example, going through all 
weather conditions for a dry road, then a gravel road etc.\\

\noindent
The following table is the result of the brainstorming session, and only 
contains identified hazards.  Hazards were identified on various surfaces,
however this number would increase due to the number of similar conditions that
a hazard can occur in, or likewise a failure can occur on one or more wheels.
Therefore all hazards identified are the generic failure which \emph{can} occur in
\emph{any possible} condition.  This is to isolate all the possible hazards that
affect the system reguardless of what conditions, in the name of improving safety.\\

\noindent
This investigation assumed that drivers of the car would be familier with road laws
and understand how to drive in various driving conditions (ie snow chains for tyres
when driving in snow etc.), however leeway was given when doing this as occasionly
circumstances occured where a hazard may arise (for example, even when driving 
cautiously on a road with black ice, a child may come in front of the car) thereby
creating a critical safety issue.\\


\begin{table}[h!tbp]
\begin{center}
\begin{tabular}{|l|l|}
\hline
	& textbf{Identified Hazards}		\\
\hline
1	&	Car skids when brakes are applied.*\\
2	&	Brakes engage too slowly when brakes applied.*\\
3	&	Brakes engage too hard and fast when brakes applied (jerkish).*\\
4	&	Brakes engaged but brakes not applied.*\\
5	&	Brakes dont engage when brakes applied.*\\
6	&	Brakes engage too soft when brakes applied.*\\
7	&	Brakes engage too much when brakes applied.*\\
8	&	Brakes dont release when brakes are not applied.*\\
9	&	Brakes release when brakes not applied.*\\
10	&	Brakes release too slow when brakes are being released.*\\
11	&	Brakes release too soft when brakes are being released.*\\
12	&	Brakes release too hard when brakes are being released.*\\
\hline
\end{tabular}
\end{center}
\caption{Results of Preliminary Hazard Identification session. * indicates 
that a hazard can occur on one or more wheels}
\label{fig:PHITable}
\end{table}

\newpage
\subsection{Preliminary Hazard Analysis}
\noindent
The following table is the hazard log related to the identified hazards of
table 2.  This table was created with reference to the risk classes
according to IEC65A (defined in lecture notes). Severity's and Frequency's
are also exracted from the same standard (again refer to lecture notes).\\



\begin{table}[h!tbp]
\begin{center}
\begin{tabular}{|l|l|}
\hline
\textbf{Reference}&		1\\
\textbf{Hazard}&	Car skids when brakes are applied.\\
\textbf{Possible Cause}&	Environmental factors, ABS failure\\
\textbf{Severity}&	Catastrophic\\
\textbf{Uncontolled Risk}&	I\\
\textbf{Estimated Tolerable Risk}&	III\\
\hline
\hline
\textbf{Reference}&		2\\
\textbf{Hazard}&	Brakes engage too slowly when brakes applied.\\
\textbf{Possible Cause}&	hydrolic line failure, actuator failure\\
\textbf{Severity}&	Critical/Catastrophic (depending on amount of failure)\\
\textbf{Uncontolled Risk}&	II\\
\textbf{Estimated Tolerable Risk}&	III\\
\hline
\hline
\textbf{Reference}&		3\\
\textbf{Hazard}&Brakes engage too hard and fast when brakes applied (jerkish).\\
\textbf{Possible Cause}&	hydrolic line failure, actuator failure\\
\textbf{Severity}&	Marginal\\
\textbf{Uncontolled Risk}&	III\\
\textbf{Estimated Tolerable Risk}&	III\\
\hline
\end{tabular}
\end{center}
\caption{Results of Preliminary Hazard Analysis: Hazard Log}
\label{fig:HazardLogTable1}
\end{table}

\begin{table}[h!tbp]
\begin{center}
\begin{tabular}{|l|l|}
\hline
\textbf{Reference}&	4\\
\textbf{Hazard}&	Brakes engaged but brakes not applied.\\
\textbf{Possible Cause}&	pressure sensor failure, actuator failure\\
\textbf{Severity}&	Marginal\\
\textbf{Uncontolled Risk}&	III\\
\textbf{Estimated Tolerable Risk}&	III\\
\hline
\hline
\textbf{Reference}&	5\\
\textbf{Hazard}&	Brakes dont engage when brakes applied.\\
\textbf{Possible Cause}& pressure sensor failure, actuator failure\\
\textbf{Severity}&	Catastrophic\\
\textbf{Uncontolled Risk}&	I\\
\textbf{Estimated Tolerable Risk}&	III\\
\hline
\hline
\textbf{Reference}&	6\\
\textbf{Hazard}&	Brakes engage too soft when brakes applied.\\
\textbf{Possible Cause}&	pressure sensor failure, hydrolic failure, actuator failure	\\
\textbf{Severity}&	Critical\\
\textbf{Uncontolled Risk}&	II\\
\textbf{Estimated Tolerable Risk}&	III\\
\hline
\end{tabular}
\end{center}
\caption{Results of Preliminary Hazard Analysis: Hazard Log}
\label{fig:HazardLogTable2}
\end{table}


\begin{table}[h!tbp]
\begin{center}
\begin{tabular}{|l|l|}
\hline
\textbf{Reference}&	7\\
\textbf{Hazard}&	Brakes engage too much when brakes applied.\\
\textbf{Possible Cause}&pressure sensor failure, hydrolic failure, actuator failure	\\
\textbf{Severity}&	Marginal\\
\textbf{Uncontolled Risk}&	III\\
\textbf{Estimated Tolerable Risk}&	III\\
\hline
\hline
\textbf{Reference}&	8\\
\textbf{Hazard}&	Brakes dont release when brakes are not applied.\\
\textbf{Possible Cause}&	pressure sensor failure, actuator failure, hydolic failure\\
\textbf{Severity}&	Marginal\\
\textbf{Uncontolled Risk}&	III\\
\textbf{Estimated Tolerable Risk}&	III\\
\hline
\hline
\textbf{Reference}&	9\\
\textbf{Hazard}&	Brakes release when brakes are applied.\\
\textbf{Possible Cause}&	pressure sensor failure, actuator failure, hydolic failure\\
\textbf{Severity}&	Catastrophic\\
\textbf{Uncontolled Risk}&	I\\
\textbf{Estimated Tolerable Risk}&	III\\
\hline
\hline
\end{tabular}
\end{center}
\caption{Results of Preliminary Hazard Analysis: Hazard Log}
\label{fig:HazardLogTable3}
\end{table}

\begin{table}[h!tbp]
\begin{center}
\begin{tabular}{|l|l|}
\hline
\textbf{Reference}&	10\\
\textbf{Hazard}&	Brakes release too slow when brakes are being released.\\
\textbf{Possible Cause}& pressure sensor failure, actuator failure, hydolic failure	\\
\textbf{Severity}&	Critical\\
\textbf{Uncontolled Risk}&	II\\
\textbf{Estimated Tolerable Risk}&	III\\
\hline
\hline
\textbf{Reference}&	11\\
\textbf{Hazard}&	Brakes release too soft when brakes are being released.\\
\textbf{Possible Cause}&	pressure sensor failure, actuator failure, hydolic failure\\
\textbf{Severity}&	Marginal\\
\textbf{Uncontolled Risk}&	IV\\
\textbf{Estimated Tolerable Risk}&	IV\\
\hline
\textbf{Reference}&	12\\
\textbf{Hazard}&	Brakes release too hard when brakes are being released.\\
\textbf{Possible Cause}&	pressure sensor failure, actuator failure, hydolic failure\\
\textbf{Severity}&	Marginal\\
\textbf{Uncontolled Risk}&	III\\
\textbf{Estimated Tolerable Risk}&	III\\
\hline
\end{tabular}
\end{center}
\caption{Results of Preliminary Hazard Analysis: Hazard Log}
\label{fig:HazardLogTable4}
\end{table}

\subsection{Justification of estimates for hazards}
\noindent
Justification for the hazards defined in the hazard log are defined below.

\begin{enumerate}
\item A car could skid due to environmental conditions such as bald tyres,
	ice on the road, or any other conditions.  Therefore for the car to skid
	other than environmental conditions would mean ABS failure.  Since this
	is a safety investigation into the braking system, it would be catastrophic
	not only in terms of reputability, but potential loss of life, from ABS
	failure.  This was therefore estimated at being an intollerable risk, 
	which must be lowered significantly, to at least a level III class.

\item If the brakes dont come on fast enough, potential loss of life may occur
	depending on the speed involved.  This risk can therefore vary upon speed
	and circumstance, but it is still a critical risk nonetheless.  It was 
	estimated that this risk is undesirable and a reduction to a class III,
	would be well worth the effort.

\item Brakes being jerky, might be a result of build up in the hydrolic pipes
	or an actuator failure.  While these are undesirable, they dont constitute
	more than a level III risk as it is tolerable for the brakes to be jerky,
	as long as it does its job.  Therefore this risk wouldnt be worth  the 
	reduction, so it would be smart to leave this risk as is (tolerable).

\item If the brakes are on, but they arnt being applied, this can present a 
	serious problem.  While this risk is serious, it does not create a major
	safety hazard, as accidents can still be avioded.  Therefore this risk is
	marginal, in that this risk is tolerable.  Any further improvement in
	risk class would not exceed the improvement gained.

\item Brakes not engaging when the brakes are applied constitutes the most 
	serious risk out of all identified.  This risk is intolerable, and must
	be ensured that is becomes at least a level II class.  However,
	it would be desirable from this risk to never happen, or occur remotly,
	therefore it would be better for this risk to become a level III risk.

\item Brakes engaging too soft may be the result of environmental conditions
	such as warn disc brakes, etc. but this risk would be marginal, as the
	risk is tolerable.  Any more risk reduction wouldnt exceed the improvement
	gained.

\item Again, this hazard is only a marginal risk, and is any amount of time
	spend reducing the risk of this happening would be great for little or
	no gain. (see hazard 3 for similarity).

\item Again, this is simlar to hazard 4 in terms of failure, tolerable and
	intolerable risk.

\item This is a major hazard rated at class I for obvious reasons.  To reduce
	this risk would be well worth the effort, as described in events described 
	eariler.

\item Brakes releasing too slow is again only a marginal risk class.
	Similarities between this hazard and hazard 2 can be drawn, in terms
	of dealing with its severity, and its primary causes.

\item This hazard is similar again to hazard 6 in both its causes, uncontrolled
	risk and tolerable risks.

\item This hazard is similar to hazard 3 in both its causes, uncontrolled risk
	and tolerable risks.

\end{enumerate}

\newpage
\section{Fault Tree Analysis}

\noindent
Fault tree analysis was performed on the top three most critical
hazards (in terms of risk classes),  to determine the possible causes of those
hazards.\\

\noindent
The top three identified hazards are:
\begin{enumerate}
	\item Reference number: 5
	\item Reference number: 1
	\item Reference number: 9
\end{enumerate}

\noindent
Due to the size of the fault trees they have been hand drawn, and placed as 
an appendix to this document (See appendix A).\\

\newpage
\section{Conclusion}
The findings of this Investigation show that the safety critical computer assisted
braking system proposed, does not meet this investigations expectations of safety 
and its design should be further explored before being implemented on testing
rigs or prototype vehicles.\\

\newpage
\appendix
\section{Fault Tree Analysis Diagrams}
\noindent
These diagrams are hand drawn and represent the three top hazards identified. 

\end{document}
